\documentclass{article}
\usepackage{graphicx}
\usepackage{amsmath}
\usepackage{amsfonts}
\usepackage{amssymb}
\usepackage{amsbsy}
\usepackage{mathabx}
\usepackage{bm}

\title{Trail diagram for plotting optimization progress}
\author{Dr. rer.nat. Matthias P. Nowak}
\date{\today}

\begin{document}
\maketitle

\begin{abstract}
Trail diagrams visualize how an optimization algorithm traverses the parameter space over time.
This work introduces the core idea, sketches how to construct the diagram from intermediate
solutions, and outlines how it complements traditional convergence plots.
\end{abstract}

\tableofcontents
\listoffigures
\listoftables

\section{Introduction}
This paper introduces a new diagram that can give ideas how the optimization walks through the parameter space, which might be high dimensional like in current AI models.

\subsection{Challenge}
Optimizing problems in two dimensions has the additional advantage that one could plot the steps taken by the algorithm and identy short comings.
Even then, convergence leads to the step size going over several magnitudes, hence plotting the steps is only helpful in simple situations.

\subsection{Basic idea}
We introduce the concept of markers in the parameter space. 
At intervals, a marker is dropped at the current position, and we maintain a certain number of markers. 
Then, at each iteration of the algorithm, we determine the distance to all markers and plot them in a plot.

\section{Results}
\subsection{Results 1}
\subsection{Results 2}
\section{Discussion}
\section{Conclusion}
\bibliographystyle{plain}
\bibliography{biblio}
\end{document}
